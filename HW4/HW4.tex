%=======================02-713 LaTeX template, following the 15-210 template==================
%
% You don't need to use LaTeX or this template, but you must turn your homework in as
% a typeset PDF somehow.
%
% How to use:
%    1. Update your information in section "A" below
%    2. Write your answers in section "B" below. Precede answers for all 
%       parts of a question with the command "\question{n}{desc}" where n is
%       the question number and "desc" is a short, one-line description of 
%       the problem. There is no need to restate the problem.
%    3. If a question has multiple parts, precede the answer to part x with the
%       command "\part{x}".
%    4. If a problem asks you to design an algorithm, use the commands
%       \algorithm, \correctness, \runtime to precede your discussion of the 
%       description of the algorithm, its correctness, and its running time, respectively.
%    5. You can include graphics by using the command \includegraphics{FILENAME}
%
\documentclass[10pt]{article}
\usepackage{amsmath,amssymb,amsthm}
\usepackage{graphicx}
\usepackage[margin=1in]{geometry}
\usepackage{fancyhdr}
\setlength{\parindent}{0pt}
\setlength{\parskip}{5pt plus 1pt}
\setlength{\headheight}{13.6pt}
\newcommand\question[2]{\vspace{.25in}\hrule\textbf{#1: #2}\vspace{.5em}\hrule\vspace{.10in}}
\renewcommand\part[1]{\vspace{.10in}\textbf{(#1)}}
\newcommand\algorithm{\vspace{.10in}\textbf{Algorithm: }}
\newcommand\correctness{\vspace{.10in}\textbf{Correctness: }}
\newcommand\runtime{\vspace{.10in}\textbf{Running time: }}
\pagestyle{fancyplain}
\lhead{\textbf{\NAME\ (\ANDREWID)}}
\chead{\textbf{HW\HWNUM}}
\rhead{\today}
\begin{document}\raggedright
%Section A==============Change the values below to match your information==================
\newcommand\NAME{Motoaki Takahashi}  % your name
\newcommand\ANDREWID{mxt323}     % your andrew id
\newcommand\HWNUM{3}              % the homework number
%Section B==============Put your answers to the questions below here=======================

% no need to restate the problem --- the graders know which problem is which,
% but replacing "The First Problem" with a short phrase will help you remember
% which problem this is when you read over your homeworks to study.

\question{1}{The First Problem} 
I drew $100^2$ points from the 2-dimensional Halton sequence, and counted the ratio of the points whose squared Euclidean norm is weakly less than 1. The estimated $\pi$ is 3.1448.



\question{2}{The Second Problem}
I have $100^2$ quadrature points in $[0, 1]\times[0, 1]$, and use a Newton-Cortes method to get an approximation of $\pi$, which is 3.1016. The weights are $1/N$, where $N$ denotes the number of quadrature points.

\question{3}{The Third Problem}
Now I use the implicit function $y=\sqrt{1-x^2}$ for the upper-right part of the unit circle. I have $100^2$ points in $[0, 1]$ from a Halton sequence, and approximate $\pi$. The estimate is 3.1422.

\question{4}{The Fourth Problem}
I have 10,000 quadrature points in $[0, 1]$, and use a Newton-Cortes method. The weights are again $1/N$, where $N$ denotes the number of quadrature points. This time I use the implicit function to get an approximation of $\pi$. The estimate is 3.1414.

\question{5}{The Fifth Question}




\end{document}
